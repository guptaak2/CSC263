\input{"../../../preamble"}

\begin{document}

\title{CSC263-Notes-03-23-2015}

\input{"../csc263-header"}
\rhead{March 23, 2015}

\section*{Lecture 21}

\noindent $G$ is connected, undirected.

\begin{lstlisting}[mathescape]
kruskal-mst($G=(V,E), w: E \rightarrow R$)
	T $\leftarrow \{\} $
	sort edges
	for each $v$ in $V$:
		make-set(v)
	for i $\leftarrow$ 1 to $m$:
		if findset($u_i$) != findset($v_i$):
			union($u_i,v_i$)
			T $\leftarrow \cup \{e_i\}$ 
\end{lstlisting}

\subsection*{Disjoint Set ADT}

\begin{itemize}
	\item[] \texttt{make-set($x$)}: create a new set $\{x\}$ that contains only
	$x$ and makes $x$ the representative
	\item[] \texttt{findset($x$)}: returns the representative of the set that
	contains $x$
	\item[] \texttt{union($x,y$)}: puts two sets together, picks new
	representative 
\end{itemize}

\subsubsection*{Hashtable approach}

\noindent key vertex, value rep

\begin{tabular}{l @{ : } l}
	f & a \\
	a & a \\
	b & b \\
	c & b \\
	d & d \\
	e & d \\
\end{tabular} \\

\begin{tabular}{l @{$\rightarrow$} l}
	a & [a] \\
	b & [b,c] \\
	d & [d,e,f] \\
\end{tabular}

\subsubsection*{Linked list approach}

$A$ \\
\indent $B \rightarrow C$ \\
\indent $D \rightarrow E \rightarrow F$ \\

\noindent circular linked list? $\rightarrow$ for \texttt{findset($x$)} \\
$\rightarrow$ use doubly linked list insteadt, go back: 

$A$ \\
\indent $B \leftrightarrow C$ \\
\indent $D \leftrightarrow E \leftrightarrow F$ \\

\noindent \texttt{union(C,E)}: \\

\noindent doubly linked list, each item has pointer to rep \\
then point B to D $\rightarrow$ better \texttt{union}, but worse \texttt{findset}   

\end{document}





















